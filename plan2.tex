\documentclass[10pt,a4paper]{article}
\usepackage[utf8]{inputenc}
\usepackage[english]{babel}
\author{Nicholas Allio \and Daniel Thul}
\title{Concurrent Programming Project 2 Plan}
\begin{document}
\maketitle

\section{Tuple Space}

\subsection{public LocalTupleSpace()}

Tuple space is implemented using a single common Lock for every operations: in this way every operation is performed in mutual exclusion.
Tuples are stored in a Map structure where the name of the tuple is used as Key, and as value is allocated an array of all tuples with that name.

\begin{verbatim}
LockObject lock
Map<String, ArrayList<Tuple>> tuplespace
\end{verbatim}

\subsection{public String[] get(String... pattern)}

Remove tuple operation will block until the correct tuple, that matches the given pattern, is retrieved from the tuple space; once the match is intercepted, values is stored and tuple is removed from the tuple space

\begin{verbatim}
operation get(pattern)
	synchronize on lock
		while not matches(pattern)
			wait on lock
		tuple = matches(pattern)
		remove tuple from tuplespace
	return tuple
\end{verbatim}

\subsection{public String[] read(String... pattern)}

Read operation works similarly to the get operation with the only difference that is not removing the tuple once it has read it.

\begin{verbatim}
operation read(pattern)
	synchronize on lock
		while not matches(pattern)
			wait on lock
		tuple = matches(pattern)
	return tuple
\end{verbatim}

\subsection{public void put(String... tuple)}

Once a tuple is inserted in the tuple space, this method notifies all threads waiting on a lock that a new tuple has been inserted; if no match is then catch, readers will come to the wait state.

\begin{verbatim}
operation put(tuple)
	synchronize on lock
		add tuple to tuplespace
		notify all threads waiting on lock
\end{verbatim}

\section{Chat System}

The chat system uses three different kind of tuples:

\textit{Server Tuple} is a single tuple that identifies the server with the number of channels and the maximum number of row that the buffer can contain:
\begin{verbatim}
('server', num_channels, num_rows)
\end{verbatim}

\textit{Channel Tuple} 
\begin{verbatim}
('channel', channel_id, channel_name, num_listeners, oldest_message_id, newest_message_id, ...?)
\end{verbatim}

\textit{Message Tuple}
\begin{verbatim}
('message', channel_name, msg_id_per_channel, message, times_to_be_read...?)
\end{verbatim}

\textit{Writer Lock}
\begin{verbatim}
('writerlock')
\end{verbatim}

\subsection{Class ChatServer}

Number of rows that every newly connected user should read is stored in a global variable
\begin{verbatim}
integer rows
\end{verbatim}

\subsubsection{public ChatServer(TupleSpace t, int rows, String[] channelNames)}

Complex constructor initialize the tuple space posting tuples in a number equal to the size of the channels, with NumListeners, Oldest\_MessageID, NewestMessageID setted to 0; ChannelName is setted accordingly to the names given in as argument of the function , and an unique identifier is assigned to each of them.
It also posts the tuple related to the server in which it sets the number of channels and the number of the buffer's size for messages that the server must keep in memory until everyone has read them.

\begin{verbatim}
	operation constructor(tmp_rows, channel_names)
		rows <- tmp_rows
		postnote('server', length of channel_names, rows)
		integer i = 1
		for each channel_name in channel_names
			postnote('channel', i, channel_name, 0, 0, 0)
			i <- i + 1
\end{verbatim}

\subsubsection{public ChatServer(TupleSpace t)}

When a new connection with the chat server is established, only the number of rows (messages) that have to be send to the newly connected user is needed; this information can be easily retrieved by reading the \textit{ServerTuple} 

\begin{verbatim}
	operator constructor()
		readnote('server', dont_care, rows)
\end{verbatim}

\subsubsection{public void writeMessage(String channel, String message)}

Every write operation is performed one at a time in order to grant mutual exclusion, data integrity and consistency, and thread safe operations. To achieve those characteristics, a tuple is used as lock and just at the end of the function other processes have the possibility to execute that piece of code.
Write a new message in the channel means modify which are id of the oldest and newest messages and the \textit{ChannelTuple} is needed to perform this operation; before this update is completed, availability of free space to insert messages is checked and the process either is blocked if no free space is available, or the operation can successfully be completed.
During this write operation, the scenario when the buffer of old messages is full can occurs and that process has to wait for the insertion of the message; in order to avoid the stalemate of the application, channel is freed by post again the same tuple previously removed without modifications and process is then blocked on the next statement when it tries to remove the oldest message; in this way users can disconnect from the chat even if they have not read all messages in the channel. Process will be blocked on the remove operation until the oldest message will be removed by the process itself: at that time the channel will be blocked again in order to complete the insertion of the new message.
If a new write operations will occur when a writer is still blocked waiting for eliminate the oldest message, they will be blocked on the writeLock because only one writer at a time is allowed inside the write operation.

\begin{verbatim}
	operation writeMessage(channel_name, message)
	removenote('writerlock')
	integer num_listeners, newest_message_id, oldest_message_id, channel_id
	removenote('channel', channel_id, channel_name=, num_listeners, oldest_message_id, newest_message_id)
	integer num_messages <- newest_message_id - oldest_message_id
	if num_messages >= rows
		postnote('channel', channel_id, channel_name, num_listeners, oldest_message_id, newest_message_id)
		removenote('message', channel_name=, oldest_message_id=, dont_care, 0)
		removenote('channel', channel_id, channel_name=, num_listeners, oldest_message_id=, newest_message_id=)
		oldest_message_id <- oldest_message_id + 1
	postnote('message', channel_name, newest_message_id, message, num_listeners)
	newest_message_id <- newest_message_id + 1
	postnote('channel', channel_id, channel_name, num_listeners, oldest_message_id, newest_message_id)
	postnote('writerlock')
\end{verbatim}

\subsubsection{public ChatListener openConnection(String channel)}
	
It creates a new instance of \textit{ChatListener} with the given channel name.

\begin{verbatim}
	operation openConnection(channel_name)
		return new ChatListener(channel_name)
\end{verbatim}

\subsubsection{public String[] getChannels()}

It iterates over the list of channels reading the name of each of them and collecting them into an array.

\begin{verbatim}
operation getChannels
  integer num_channels
		String[] channel_names
		readnote('server', num_channels, dont_care)
		for i from 1 to num_channels
			String channel_name
			readnote('channel', i=, channel_name, dont_care, dont_care, ...)
			add channel_name to channel_names
		return channel_names
\end{verbatim}


\subsection{Class ChatListener}

\begin{verbatim}
string channel_name
integer next_message_id
\end{verbatim}

\subsubsection{public ChatListener(String channel\_name)}

When a user connects to the server, an instance of ChatListener is created and all messages in the buffer have to be send to the newly connected user. In addition to this operation, because of the new user registration at the chat, two information should be updated: number of listeners on the server is incremented and, for each message in the channel, the counter of times that each of them should be read, is incremented as well.

\begin{verbatim}	
	operation constructor(tmp_channel_name)
		channel_name <- tmp_channel_name
		integer num_listeners, newest_message_id, channel_id
		removenote('channel', channel_id, channel_name=, num_listeners, next_message_id, newest_message_id)
		for i from next_message_id to newest_message_id - 1
			string message
			integer times_to_be_read
			removenote('message', channel_name=, i=, message, times_to_be_read)
			times_to_be_read <- times_to_be_read + 1
			postnote('message', channel_name, i, message, times_to_be_read)
		num_listeners <- num_listeners + 1
		postnote('channel', channel_id, channel_name, num_listeners, next_message_id, newest_message_id)
\end{verbatim}

\subsubsection{public String getNextMessage()}

Next message to be shown is retrieved, showed and updated before post it again into the tuple space. Process will block if no messages are available since is always requiring messages from a specific channel name and a specific id for the message.
Once the message is read, the number of times that has to be read is updated.

\begin{verbatim}
	operation getNextMessage
		string message
		integer times_to_be_read
		removenote('message', channel_name=, next_message_id=, message, times_to_be_read)
		times_to_be_read <- times_to_be_read - 1
		postnote('message', channel_name, next_message_id, message, times_to_be_read)
		next_message_id <- next_message_id + 1
\end{verbatim}

\subsubsection{public String getNextMessage()}

When a user disconnects from the chat, all messages have to be updated because the number of times to be read should be decremented by one. This operation is performed by removing the channel tuple, in order to avoid every reader or writer to access the channel and make modifications in the meantime.

\begin{verbatim}
	operation closeConnection
		integer num_listeners, oldest_message_id, newest_message_id, channel_id
		removenote('channel', channel_id, channel_name=, num_listeners, oldest_message_id, newest_message_id)
		for i from next_message_id to newest_message_id - 1
			string message
			integer times_to_be_read
			removenote('message', channel_name=, i=, message, times_to_be_read)
			times_to_be_read <- times_to_be_read - 1
			postnote('message', channel_name, i, message, times_to_be_read)
		num_listeners <- num_listeners - 1
		postnote('channel', channel_id, channel_name, num_listeners, oldest_message_id, newest_message_id)
\end{verbatim}

\end{document}
